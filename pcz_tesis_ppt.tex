\documentclass[mathserif]{beamer}
\newcommand{\be}{\begin{equation}}
\newcommand{\ee}{\end{equation}}
\newcommand{\bce}{\begin{center}}
\newcommand{\ece}{\end{center}}
\newcommand{\dpa}[2]{\frac{\partial #1}{\partial #2}}
\newcommand{\ndp}[3]{\frac{\partial^#3 #1}{\partial #2 ^#3}}
\newcommand{\cdpa}[3]{\frac{\partial^2 #1}{\partial #2 \partial #3}}
\usepackage[utf8]{inputenc}
\usepackage{ragged2e}
\usepackage{etoolbox}
\usepackage{animate}
\usepackage{booktabs}
\usepackage{eulervm}
\apptocmd{\frame}{}{\justifying}{}
\usepackage[activeacute,spanish]{babel}
\setbeamertemplate{navigation symbols}{}
\setbeamertemplate{caption}[numbered]
\usetheme{default}
\setbeamertemplate{footline}[frame number]
\definecolor{cobalt}{rgb}{0.0, 0.28, 0.67}
\usecolortheme[named=cobalt]{structure}
\setbeamercolor{alerted text}{fg=blue}
\title{Simulación Multiescala de Viento Sobre Terreno Complejo Mediante el Método Embedido WRF-LES y Asimilación Variacional de Datos 4D}
\author{Pablo Andrés Cárdenas Zamorano}
\institute[Universidad Técnica Federico Santa María]
{%
  Magíster en Ciencias de la Ingeniería Mecánica,\\
  Universidad Técnica Federico Santa María\\
  \bigskip
  \begin{tabular}{ll}
  	 Profesor Guía:& Ph.D. Alex Flores Maradiaga\\
  	 Profesor Correferente:& Ph.D. Carlos Rosales Huerta\\
  	 Evaluador Externo:& Ph.D. Ricardo Muñoz Magnino \\
  \end{tabular}
 }
\date{\footnotesize Agosto, 2019}
\AtBeginSubsection[]
{
  \begin{frame}<beamer>{Outline}
    \tableofcontents[currentsection,currentsubsection]
  \end{frame}
}
\begin{document}
\begin{frame}
	\vspace{0.3cm}
	\begin{center} \includegraphics[height=1.5cm]{utfsm_logo} \end{center}
	\vspace{-0.5cm}
	\titlepage
\end{frame}

\begin{frame}{Contenidos}
	\tableofcontents
\end{frame}

\section{1. Motivación}
\begin{frame}{Motivación}{¿Por qué Predecir el Viento?}
\begin{figure}
	\centering
	\includegraphics[width=1.0\linewidth,page=30,trim={2cm 18cm 2.5cm 3cm},clip]{fig/01/Anuario-CNE-2018}
	\vspace{3mm}
	\includegraphics[width=1.0\linewidth,page=30,trim={2cm 2.5cm 2.5cm 23.1cm},clip]{fig/01/Anuario-CNE-2018}
	\vspace{3mm}
	\caption{Evolución de la matriz energética chilena. Fuente: Comisión Nacional de Energía (2018).}
\end{figure}
\end{frame}

\begin{frame}{Motivación}{¿Por qué Predecir el Viento?}
	\begin{figure}
		\centering
		\includegraphics[width=0.7\linewidth,trim={2.6cm 1.4cm 1.5cm 0.8cm},clip]{fig/02/escalas}
		\vspace{-2mm}
		\caption{Unificación de escalas en dinámica atmosférica. Fuente: Montornes et al. (2017).}
	\end{figure}
\end{frame}

\begin{frame}{Motivación}{¿Cómo Predecirlo?}
	\begin{enumerate}[a.]
		\item Extrapolación Estadística / Simulación Numérica.
		\item Modelos Meteorológicos / CFD.
		\item Correcta representación de la CLP (PBL).
		\item Turbulencia y Terreno Complejo.
	\end{enumerate}
\end{frame}

\begin{frame}{Motivación}{¿Cómo Predecirlo?}
\begin{figure}[h]
	\centering
	\includegraphics[width=0.8\linewidth,trim={1.4cm 28cm 15cm 3.4cm},clip]{fig/01/explo}
	\vspace{-4mm}
	\caption{Interfaz online del explorador eólico de la Universidad de Chile.}
	\label{fig:01_explorador}
\end{figure}
\end{frame}

\begin{frame}{Motivación}{¿Cómo Predecirlo?}
	\begin{figure}[h]
		\begin{minipage}{0.5\linewidth}
		\centering
		\includegraphics[width=1.0\linewidth,page=5,trim={5cm 9cm 2.3cm 3cm},clip]{fig/01/descrp}
		\end{minipage}%
		\begin{minipage}{0.5\linewidth}
			\centering
			\includegraphics[width=0.9\linewidth,page=5,trim={3cm 2.5cm 5cm 18.5cm},clip]{fig/01/descrp}
		\end{minipage}%
		\caption{Esquema de la sonda FONDEF ID16I10105.}
		\label{fig:01_sonda}
	\end{figure}
\end{frame}

\begin{frame}{Motivación}{¿Cómo Predecirlo?}
	\begin{figure}
		\begin{minipage}{0.5\linewidth}
			\centering
			(a)
		\end{minipage}%
		\begin{minipage}{0.5\linewidth}
			\centering
			(b)
		\end{minipage}%
		
		\begin{minipage}{0.5\linewidth}
			\centering
			\includegraphics[width=0.9\linewidth,page=3,trim={6cm 12.2cm 6cm 9.5cm},clip]{fig/01/descrp}
		\end{minipage}%
		\begin{minipage}{0.5\linewidth}
			\centering
			\includegraphics[width=0.7\linewidth,trim={0cm 0cm 0cm 0cm},clip]{fig/01/prototipo}
		\end{minipage}%
		\caption{Detalle del proyecto FONDEF ID16I10105. (a) Célula del sistema experimental de medición. (b) Prototipo en el laboratorio.}
		\label{fig:01_detalle_fondef}
	\end{figure}
\end{frame}

\section{2. Hipótesis y Objetivos}
\begin{frame}{Hipótesis y Objetivos}
	\begin{block}{Hipótesis}\justifying
		Se pueden mejorar las predicciones numéricas de viento a corto plazo sobre terreno complejo a través de simulaciones multiescala de alta resolución, LES y asimilación de datos 4D en la CLP.
	\end{block}
	\begin{block}{Objetivo Principal}\justifying
		Implementar una metodología que incorpore escalamiento dinámico de dominios, nuevas bases de datos de alta resolución, LES y asimilación de datos 4D multipunto para mejorar los resultados de modelos numéricos de viento sobre terreno complejo.
	\end{block}
\end{frame}

\section{3. Estado del Arte}
\begin{frame}{Estado del Arte}
	\begin{enumerate}[a.]
		\item Problemáticas del escalamiento dinámico.
		\item Antecedentes de turbulencia atmosférica y terreno complejo.
		\item Desafios de la alta resolución en terreno complejo.
		\item Contexto de la asimilación de datos.
	\end{enumerate}
\end{frame}

\begin{frame}{Estado del Arte}{Problemáticas del escalamiento dinámico}
	\begin{figure}
		\centering
		\includegraphics[width=0.7\linewidth,trim={2.6cm 1.4cm 1.5cm 0.8cm},clip]{fig/02/escalas}
		\vspace{-2mm}
		\caption{Unificación de escalas en dinámica atmosférica. Fuente: Montornes et al. (2017).}
	\end{figure}
\end{frame}

\begin{frame}{Estado del Arte}{Problemáticas del escalamiento dinámico}
	\begin{figure}
		\centering
		\includegraphics[width=0.6\linewidth,trim={2.67cm 1.35cm 3.1cm 2.25cm},clip]{fig/02/grid}
		\caption{Idealización de los distintos tamaños de vórtices dentro de un dominio en la zona gris de la turbulencia. Fuente: Montornes et al. (2017).}
		\label{fig:02_grid_vortex}
	\end{figure}
\end{frame}

\begin{frame}{Estado del Arte}{Problemáticas del escalamiento dinámico}
	\begin{figure}
		\centering
		\includegraphics[width=0.9\linewidth,trim={2cm 3.0cm 1.5cm 11.5cm},clip]{fig/02/terra_inc}
		\caption{Espectro de energía cinética turbulenta multiescala. Fuente: Warner (2010).}
		\label{fig:02_terra_inc}
	\end{figure}
\end{frame}

\begin{frame}{Estado del Arte}{Turbulencia Atmosférica y Terreno Complejo}
	\begin{itemize}
		\item Modelos Lineales (Jackson y Hunt 1975, Mason y Sykes 1979)
		\item No lineales: 2D (Taylor 1977), RANS (Launder y Spalding 1974)
		\item LES: Desde los 90s se viene desarrollando para la CLP.
		\item Simulaciones Askeverin (2009) y cerros sinusoidales (2001).
	\end{itemize}
\end{frame}

\begin{frame}{Estado del Arte}{Alta Resolución y Terreno Complejo}
	\begin{itemize}
		\item Aspectos Computacionales.
		\item Aspectos Numéricos.
		\begin{itemize}
			\item Precisión
			\item Estabilidad
			\item Difusión Numérica
			\item Coordenadas
			\item Benchmarking
		\end{itemize}
		\item Parametrización de CLP.
		\item Inicialización y Datos de Entrada.
	\end{itemize}
\end{frame}

\begin{frame}{Estado del Arte}{Alta Resolución y Terreno Complejo}
	\begin{figure}[h!]
		\centering
		\vspace{0.2cm}
		\includegraphics[width=0.75\linewidth,trim={2.6cm 13.5cm 9.2cm 9cm},clip]{fig/02/coordinates}
		\caption{Comparación entre las coordenadas usuales sigma (arriba) y el método de frontera inmersa (abajo). Fuente: Arnold et al. (2010).}
	\end{figure}
\end{frame}

\begin{frame}{Estado del Arte}{Alta Resolución y Terreno Complejo}
	\begin{itemize}
		\item Aspectos Computacionales.
		\item Aspectos Numéricos.
		\begin{itemize}
			\item Precisión
			\item Estabilidad
			\item Difusión Numérica
			\item Coordenadas
			\item Benchmarking
		\end{itemize}
		\item Parametrización de CLP.
		\item Inicialización y Datos de Entrada.
	\end{itemize}
\end{frame}

\begin{frame}{Estado del Arte}{Contexto de la Asimilación de Datos}
	\begin{itemize}
		\item aaa
		\item bbb
		\item ccc
	\end{itemize}
\end{frame}












%MARCO TEÓRICO
\section{4. Marco Teórico}
\begin{frame}{Marco Teórico}
	\begin{enumerate}[a.]
		\item Leyes fundamentales de un Fluido.
		\item Ecuaciones que rigen la Dinámica Atmosférica.
		\item Turbulencia.
		\item Fundamentos de Capa Límite Atmosférica.
		\item Simulación de Grandes Vórtices.
		\item Asimilación de Datos.
	\end{enumerate}
\end{frame}

\section{4. Marco Teórico}
\begin{frame}{Marco Teórico}{Leyes Fundamentales de un Fluido}
	Conservación de Masa:
	\be \partial_t \rho + \partial_i(\rho u_i) = 0 \ee
	Conservación de Momentum:
	\be \rho d_t u_i = \rho g_i + \partial_j\sigma_{ij} \ee
	Conservación de Energía:
	\be \rho d_t\left( e+ K \right) = u_i\rho g_i + \partial_j(u_i \sigma_{ij}) - \partial_j q_i \ee
	Ecuación de Estado:
	\be p = f(\rho,T) \ee
\end{frame}

\section{4. Marco Teórico}
\begin{frame}{Marco Teórico}{Ecuaciones de Dinámica Atmosférica}
Ecuaciones Primitivas:
\small
	\begin{align}
	d_t u &= \frac{uv\tan\psi}{a}-\frac{uw}{a}-\frac{1}{\rho}\partial_x p - 2\Omega_e(w\cos\psi - v\sin\psi) + F_{rx}\\
	d_t v &= -\frac{u^2\tan\psi}{a}-\frac{uw}{a}-\frac{1}{\rho}\partial_y p - 2\Omega_e u\sin\psi + F_{ry}\\
	d_t w &= \frac{u^2 + v^2}{a}-\frac{1}{\rho}\partial_z p + 2\Omega_e u\cos\psi -g + F_{rz}\\
	\partial_t T &= -u\partial_x T -v\partial_y T + (\gamma-\gamma_d)w+\frac{1}{C_p}d_t H\\
	d_t \rho &= -\rho(\partial_i u_i)\\
	d_t q_v &= Q_v\label{03_eq:humedad}\\
	p &= \rho R T.
	\end{align}
\normalsize
\end{frame}



















\section{5. Modelo WRF}
\begin{frame}{Modelo WRF}
\end{frame}

\section{6. Metodología}
\begin{frame}{Metodología}
	\begin{figure}[h!]
		\centering
		%\vspace{0.2cm}
		\includegraphics[width=1\linewidth,clip]{tesis_experimentos}
		\caption{Diagrama de experimentos realizados.}
	\end{figure}
\end{frame}

\section{7. Resultados}
\begin{frame}{Resultados}
\end{frame}

\section{8. Conclusiones}
\begin{frame}{Conclusiones}
\end{frame}

\begin{frame}{Agradecimientos}
\end{frame}


\begin{frame}
	\vspace{0.3cm}
	\begin{center} \includegraphics[height=1.5cm]{utfsm_logo} \end{center}
	\vspace{-0.5cm}
	\titlepage
\end{frame}

\end{document} 